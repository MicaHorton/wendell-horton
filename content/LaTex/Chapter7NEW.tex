\documentclass[a4paper,openany,12pt]{book}
\bibliographystyle{plainnat}
\usepackage{mathtools}
\usepackage{url}
\usepackage{epsfig}
\usepackage{graphicx}
\usepackage{bm}
\usepackage{undertilde}
\usepackage[compact]{titlesec}
\usepackage{setspace}
\usepackage{morefloats}
\usepackage{float}
\def\margins{\textwidth 6.9in
             \evensidemargin 0.0in
             \oddsidemargin 0.0in
             \marginparwidth -5in
             \textheight 8.5in
             \topmargin -.5in
             \topskip 0.0in}
\margins
\def\dsp{{\displaystyle}}
\raggedbottom
\titlespacing*{\section}{0pt}{5.5ex plus 1ex minus .2ex}{4.3ex plus .2ex}
\titlespacing*{\subsection}{0pt}{5.5ex plus 1ex minus .2ex}{4.3ex plus .2ex}
\titlespacing{\subsubsection}{2pt}{*2}{*2}
\linespread{1}
\setlength{\parindent}{25pt}
\def\Re{\mathop{\rm Re}\nolimits}
\def\Im{\mathop{\rm Im}\nolimits}
\def\chix{\raise.5ex\hbox{$\chi$}}
\def\etal{{\it et~al.}}
\def\calo{\cal O}
\def\call{{\cal L}}
\def\nms{\mathsurround=0pt}
\def\gtsim{\mathrel{\mathpalette\oversim>}} % greater than or sim.
\def\ltsim{\mathrel{\mathpalette\oversim<}} % less than or sim.
\def\oversim#1#2{\lower 2pt\vbox{\baselineskip 0pt \lineskip 1pt
    \ialign{$\nms#1\hfil##\hfil$\crcr#2\crcr\sim\crcr}}}
\def\gtequal{\mathrel{\mathpalette\overequal>}} % greater than or equal.
\def\ltequal{\mathrel{\mathpalette\overequal<}} % less than or equal.
\def\overequal#1#2{\lower 2pt\vbox{\baselineskip 0pt \lineskip 1pt
    \ialign{$\nms#1\hfil##\hfil$\crcr#2\crcr=\crcr}}}
\def\m@th{\mathsurround=0pt}
\def\n@space{\nulldelimiterspace=0pt \m@th}%1
\def\biggg#1{{\mbox{$\left#1\vbox to 20.5pt{}\right.\n@space$}}}%2
\def\Biggg#1{{\mbox{$\left#1\vbox to 23.5pt{}\right.\n@space$}}}%3
\def\Bigggg#1{{\mbox{$\left#1\vbox to 40pt{}\right.\n@space$}}}%4
\setcounter{chapter}{6}
\begin{document}

%Chap 7
\chapter{Future for Fusion Power Machines}

%7.1
\section{Building Small Stars for Producing Power by ``Burning" Hydrogen Isotopes}

The effort to build plasma machines reproducing the methods of stars for producing power by ``burning" hydrogen isotopes is progressing worldwide with an increasing number of machine designs. The tokamak approach used in the ITER machine is currently be tested in an international partnership in new facility on a mesa plateau near Aix en Provence, France. Next to the ITER site is the classified French nuclear power laboratory that builds the fission reactors used around the world in many countries.

%7.2
\section{The ITER Machine Will Show ``The Way"}

ITER is a name adopted from Latin to describe ``the Way". The meaning is that the International Toroidal Energy Reactor is the way to produce the first nuclear fusion power reactor. Mankind is attempting to reproduce mother nature's way to vast, clean nuclear energy sources with plasmas confined by magnetic fields in large toroidal vacuum vessels. The search for best confinement vessels started as classified work under the Sherwood project in the USA and top secrete classified efforts in the Soviet Union and England until an international declassification agreement was reached in 1970s.

%7.3
\section{Solar Nuclear Burning Fusion Plasmas}

Solar nuclear burning fusion plasmas produce helium as the ash and high--energy neutrons that rapidly decay or combine to form heavier elements. The process produces a low level of radioactive waste in contrast to fission nuclear reactions.

%7.4
\section{Complex Magnetic Fields Configurations}

In nature, like in our Sun and other stars, complex magnetic fields configurations develop from the electromagnetic dynamics of the high temperature rotating plasmas and the pull of gravity. In the laboratory scientists and engineers are developing complex magnetic geometries to achieve the confinement and burning of the low mass nuclear elements from lithium, beryllium, and boron to form carbon as the exhaust materials. The fusion nuclear reactions are hundreds of times stronger than the fission nuclear reactions and produce a much safer exhaust --- helium --- than the fission nuclear reactors that produce. 

Nuclear fusion machines are complicated owing to the need to isolate the solar temperature plasmas and even higher temperature plasmas --- from the machine walls and to absorb the high--energy neutrons admitted in the nuclear fusion reactions.

%7.5
\section{Isolating the Burning Plasma from the Walls}

Isolating the burning plasma from the walls of the confinement machine remains a challenging issue. Producing the controlled burning fusion of plasma is a critical problem for the next generation of physicists and engineers. Complex magnetic field vessels with special walls and plasma heating waves and beams will be needed and are being designed and tested in broad international and national activities. Safely confining and controlling the nuclear fusion ``burning" plasma has proven to be a century--long research project. The exhaust products are high--energy alpha particles and neutrons.

%7.5.1
\subsection{Alpha particle plasma heating rates}

Important parameters in the energy balance of thermonuclear plasmas are the alpha particle heating rates and the division of the alpha particle energy that is transferred between ions and electrons. Here we review the collective mechanisms for the transfer of alpha energy to the plasma fuel and point out that these collective wave mechanisms are much faster than the classical collisional transfer rates. The collisional alpha particle of mass $m_a$ and energy $E_\alpha$ slowing--down time is $\tau_s\simeq 3,m_\alpha m_e v^3_e/[(2\pi)^{1/2} m_e z_\alpha^2 e^4\ell n\,\Gamma]$ and is of order one second for $n_e = 10^{14}\,\rm cm^{-3}$ and $T_e\approx m_e v_e^2\approx 10$-20$\,$KeV.

Here we analyze the alpha particle energy distributions peaked around the source energy of $2.5\,$MeV there are electromagnetic instabilities with scale $k^{-1}\simeq c/\omega_{pi}$ and electrostatic instabilities with scale $k^{-1}\gtsim v_i/\omega_{ci}=\rho_i$ that grow on the ion cyclotron period time scale and lead to anomalous diffusion equations for the particle distribution functions. We review results for the electrostatic modes and the electromagnetic modes. 

The stability of a plasma with Maxwellian ions and electrons and the high--energy alpha particle distribution $f_\alpha(v^2)$ having various degrees of peaking about the source energy $E_\alpha = 1/2m_\alpha v^2 = 3.5\,$MeV is a function of $E_\alpha/T_e$ and $n_\alpha/n_e$. The alpha particle distributions are parameterized by $(p,n)$ as follows: 
\begin{equation}
f_\alpha^n\left(v^2\right)={p^{2n+3}\over 2\pi\Gamma(n+3/2)}v^{2n} e^{-p^2 v^2}\label{E7.1}
\end{equation}
where $E_\alpha=m_\alpha(2n+3)/4p^2$. In the limit $n\to\infty$ with $n/p^2$ finite, the distribution is monoenergetic at $E_\alpha$. 

For the electrostatic dispersion relation$\varepsilon(\omega)$ with Maxwellian ions and electrons and calculating the fluctuating resonant alpha particle density contribution, $\rm Im\,\widetilde\rho_\alpha$, $\rho_\alpha(k,\omega)$ for low alpha--particle concentration $n_\alpha/n_e < 1$ one obtains the dispersion relation 
\begin{eqnarray}
&&k^2\varepsilon_{\mathbf{k}}(\omega)=i\pi\omega\omega^2_{p\alpha}\sum_\ell\int dv j^2_\ell\delta\left(\omega-k_\| v_\|-\ell\omega_{c\alpha}\right){\partial f_\alpha\over\partial v^2/2}\nonumber\\
&&\mbox{}\quad {i\pi\omega\omega^2_{p\alpha} p^3\over|k_\||\Gamma(n+3/2)}\sum_\ell e^{-p^2 c^2_\ell}\int_0^\infty 2y\, dy\, e^{-y^2}\nonumber\\
&&J^2_\ell\left({l_{\|y}\over p\omega_{c\alpha}}\right)\left(n-p^2 c^2_\ell-y^2\right)\left(p^2 c^2_\ell=y^2\right)^{n-1}\label{E7.2}
\end{eqnarray}
where $c_\ell=(\omega-\ell\omega)/k_\|$.

First we consider the mode at the deuterium cyclotron frequency $\omega_{cD}=\omega_{c\alpha}=\omega_{ci}$ for wavelengths $k_\perp\rho_i < 1$. The mode frequency is $\omega_{\mathbf k}\cong\omega_{ci}(1-1/2m_e k^2_\perp/m_i k^2)$ for $k_\perp\rho_i<k_\| v_e/\omega_{ci}<1$ and $\omega_{\mathbf{k}}\cong\omega_{ci}(1-1/2 k^2_\perp\rho_i^2 T_e/T_i)$ for $k_\| v_e/\omega_{ci}>1$. For $k_\perp\rho_\alpha=k_\perp v_\alpha/\omega_{c\alpha}>1$ with $v_\alpha=p^{-1}$ the alphas with $v_\perp<v_\alpha$ dominate the resonant contribution to $\widetilde\rho_\alpha$ and for these particles $(v_\perp< v_\alpha)$ the distribution in $v_\|$ is double humped with the resonance destabilizing for $c_{\ell=1}| < v_\alpha$. For this gyrofrequency mode the $\alpha$--particle contributions for the mildly--peaked $n=1$ distribution is
\begin{equation}
k^2\varepsilon_{\mathbf{k}}=i{4\omega^2_{ci}\omega^3_{p\alpha}p^4\over 3|k_\||k_\perp|}\left({1\over 2-p^2 c^2_1}\right)\label{E7.3}
\end{equation}
where $c_1=(\omega_{\mathbf{k}}-\omega_{ci})/k_\|=\delta\omega_{\mathbf{k}}/k_\|$. 
The growth rate reaches a maximum in the $k_\| v_e/\omega_{ci}<1$ region between the curves $2v_i<c_1\ltsim 0.5 v_\alpha$ and where $k_\perp\rho_i\gtsim v_i/v_\alpha$. The maximum growth rate is of order $\gamma/\omega_{ci}\simeq 0.2(n_\alpha/n_e)(m_e/M_i)^{1/3}$.

We model the wavenumber spectrum as 
\begin{equation}
I\left(k_\perp, k_\|\right)=I_{\rm max}\ \exp\left[-{1\over 2}\left(k_\perp-\overline k_\perp\right)^2/\Delta k^2_\perp-{1\over 2}\left(k_\|-\overline k_\|\right)^2/\Delta k^2_\|\right].\label{E7.4}
\end{equation}
The correlation function along the particle orbit is
\begin{equation}
C_{\mathbf{k}}(\bm{v},t)(\mathbf{v},t)\cong\left<\varphi^2\right>\sum_n J^2_n\left({\overline k_\perp v_\perp\over\omega_c}\right) e^{-1/2}\Delta\omega^\alpha_{\mathbf{k} t^2}\cos\left[\left(\omega_{\overline{\mathbf{k}}}-\overline k_\| v_\|-n\omega_c\right) t\right]\label{E7.5}
\end{equation}
where $\Delta\omega^2_{\mathbf k}=\Delta k^2_\| v^2_\|+\Delta k^2_\perp\left(d\omega_{\overline{\mathbf k}}/
d\overline k_\perp\right)^2$.\\[3pt]

The velocity diffusion equation is
%
\begin{equation}
{\partial f_j\over\partial t}={\partial\over\partial\bm{v}}\left(\utilde{\utilde{D}}\cdot{\partial f_j\over\partial\mathbf{v}}\right)\label{E7.6}
\end{equation}
where
\begin{equation}
\utilde{\utilde D}=\left({e_j\over m_j}\right)^2\sum_n\int_{-\infty}^0 dt\, C_{\mathbf{k}, n}(\mathbf{v}, t)\left(\begin{array}{cc} {n^2\Omega^2\over v^2_\perp}  &\qquad{n\Omega k_\|\over v_\perp}\\[10pt]
{n\Omega\over v_\perp}  &\qquad k^2_\|\end{array}\right).\label{E7.7}
\end{equation}
For the ions the velocity diffusion reduces to 
\begin{equation}
D^i_\perp\simeq\left({\pi\over 2}\right)^{1/2}\ {e^2\left<\varphi^2\right>\overline k^2_\perp\over m_i^2\Delta\omega_{\mathbf k}} 
e^{-1/2}\left(\delta\omega_{\overline{\mathbf k}}-\overline k_\| v_\|\right)^2\big/\Delta\omega^2_{\mathbf{k}}.\label{E7.8}
\end{equation}

From Eq.~(\ref{E7.6}) one finds that the fast ions with $v_\|=\delta\omega_{\overline{\mathbf{k}}}/\overline k_\|$ diffuse or heat at the rate
$$\tau^{-1}_{Di}=D^i_\perp/v^2_\perp=\omega_{ci}\ {v^2_i\over v^2_\perp v^2_\alpha}\ {e^2\left<\varphi^2\right>\over T^2_i}$$
and the thermal ions with $v_\|\simeq v_i<\delta\omega_{\overline{\mathbf{k}}}/\overline k_\|$ heat at a rate reduced by $\exp\left[{-1\over 2} (\delta\omega_{\overline{\mathbf{k}}}/\Delta\omega_{\mathbf{k}})^2\right]$.

The alpha particles diffuse and cool at the rate
$$\tau^{-1}_{D\alpha}={D_\alpha\over v^2_\alpha}\simeq\omega_{ci}\left({\omega_{ci}\over\overline k_\perp v_\alpha}\right) {v^4_i\over v^4_\alpha}\ {\left<\varphi^2\right>\over T^2_i}.$$
These collective heating rates compare to the classical collisional slowing--down time $\tau_\alpha$ as
\begin{equation}
{\tau_s\over\tau_\alpha}\simeq{e^2\left<\varphi^2\right>\over T^2_i}\ {\omega_{ce}\over\omega_{pe}}\left(n_e\lambda^3_{De}\right).\label{E7.9}
\end{equation}
%
%7.5.2
\subsection{Drift waves driven by ion pressure gradients}

Ion pressure gradient--driven drift modes are analyzed for their parametric dependence on the magnetic shear, the toroidal aspect ratio, and the pressure gradient using the ballooning toroidal mode theory [\emph{Horton, et al.} (1981)]. An approximate formula for the anomalous ion thermal conductivity is derived for the turbulent regime. In view of the 
high ion temperatures produced by powerful auxiliary heating in tokamaks, it is important to re--examine the ion pressure gradient driven drift modes along with their associated anomalous thermal transport $\chix_i$. In this section we analyze the mode structure of these pressure gradient driven modes using the methods developed for studying toroidal drift modes [\emph{Choi and Horton} (1980), \emph{Hastie, et al.} (1979), \emph{Frieman, et al.} (1980), \emph{Chen and Cheng} (1980), \emph{Horton, et al.} (1978)]. The parametric variation of the turbulence with toroidal curvature, magnetic shear and the pressure gradient of the unstable mode characteristics, such as their angular width, average radial and parallel wavenumbers, is used for their identification from the laboratory plasmas fluctuation measurements.

%7.5.3
\subsection{Magnetic fluctuations and electron transport in the ionosphere}

The scientific activity will be associated with a theoretical description of the generation of magnetic coherent structures and turbulence in the ionosphere. These structures mainly determine the turbulent spatial transport and heating of the electrons in these regions. The plasma data shows signatures of coherent structures that are a combination of magnetic waves and drift waves forming 3D structures. The nonlinear equations will be solved for the formation of the plasma structures. 

One of the major problems of atmospheric and laboratory physics refers to the description of coherent structures in the ionosphere which are similar but more complex that those of the neutral gas atmosphere. By using nonlinear 3D computer simulations, we construct models for the ionosphere plasma dynamics and predictions of disruptive events. 

Drift waves were originally discovered by \emph{D'Angelo and Motley} (1963) and \emph{Chen} (1965a,b, 1966, 1967); their properties were documented in detail by \emph{Hendel, et al.} (1968) in 
$Q$--machine plasmas. The first detailed explanation of the nonlinear oscillations measured by Hendel's experiments is given by \emph{Hinton and Horton} (1971). In that work, the square of the density gradient drives the growth rate of the waves through the electron resistivity and parallel thermal diffusivity in a plasma where ion collisional viscosity governs the high wavenumber damping. From these drift wave experiments it became clear that the Bohm diffusion [\emph{Bohm, et al.} (1949)] was produced by the large convection in the drift waves driven by the plasma density gradients. The drift wave instability plays a key role in the turbulent transport of magnetized fusion plasmas, which makes the literature on this subject extensive. For a review see, for instance, \emph{Horton} (1999) and references therein. This instability, sometimes called universal instability, is present in most laboratory plasmas due to the necessary presence of density and temperature gradients in confined plasmas. However, the experimental study of drift wave instabilities and the associated turbulence and turbulent transport are not simple problems. This difficulty comes from the fact that in order to obtain good plasma confinement a complex toroidal magnetic geometry is necessary. In addition, a complete experimental description of the fluctuations is not always possible either because of the high temperatures of fusion--grade plasmas or access limitations due to configurational constraints in the plasma geometry in the confinement devices. 

The Helimak is one of the classic basic plasma experiments which shows the characteristics of a magnetic fusion plasma in a simpler geometry and with better diagnostics than are possible in major fusion confinement devices. The Helimak, now operating with numerous upgrades in China, is a finite realization of the cylindrical--sheared slab plasma often used in theoretical calculations of plasma turbulence. This correspondence makes possible the comparison of well--understood theoretical and numerical models with experimental data. The additions of magnetic curvature and shear are the minimal additions to the slab model required to introduce the effects of a magnetic confinement geometry. The field line curvature generates a charge--dependent drift of the guiding centers which separates electrons from ions. This charge separation is also mathematically equivalent to the one that occurs from the gravitational field on the surface of the sun and Jupiter. Thus, the experiment can simulate the Rayleigh--Taylor instability. 

The Helimak produces a spiraling toroidal magnetic field. The dominant toroidal field is of order $0.1\,\rm T=1\,kg$ with a weaker vertical field $B_z$ which may be varied up to 10\% of the toroidal field. The field lines are thus helices as shown spiraling from bottom to top of the two meter cylindrical chamber. The helical field line length may be varied from less than $10\,$m to more than $1\,$km. The height of the vacuum vessel is $L_z = 2H = 2\,$m, the inner radius is $0.6\,$m, and the outer radius $1.6\,$m. For most conditions, all the field lines terminate at both ends of the cylinder on sets of metal plates as shown. There are two sets of plates, top and bottom, and are $180^\circ$ apart. For the steepest pitches, some field lines terminate on the vessel. Although each plate is electrically isolated, all are connected to the vacuum vessel in these experiments. The field lines impinge nearly normal to the plates, which are dotted with more than 700 surface--mounted Langmuir probes. The insert in the figure illustrates one probe tip with its ceramic insulator protruding through the plate into the plasma. 

The plasma is formed and heated with $6\,$kW of microwave power at the electron cyclotron frequency. The power is admitted through an open waveguide on the high--field side. Since the single--pass absorption is small in this experiment, the vacuum chamber forms an over--moded, low--$Q$ cavity similar to a microwave oven. In argon at a neutral density of $4\times 10^{11}\rm cm^{-3}$, typical plasma parameters are $n_e = 10^{11}\,\rm cm^{-3}$, $T_e = 10\,$eV, and $T_i = 0.1\,$eV. This configuration has a simple stable Magnetohydrodynamic (MHD) equilibrium [\emph{Parail, et al.} (1985), \emph{Zimmerman and Luckhardt} (1993), \emph{Luckhardt} (1999)] in which the charge separation from vertical drifts is largely neutralized by a return $j_\|$ and the small $j_\perp$ currents required for force balance flows to the end plates and returns through the conducting vessel. 

The configuration has been used in other experiments described in \emph{M\"uller, et al.} (2004) and \emph{Rypdal and Ratynskaia} (2005). The latter experiment also uses microwave power for plasma production and reports plasmas with similar equilibrium parameters. The Helimak experiment differs from these principally in size, having dimensions large compared with all scale lengths, including these for density and temperature gradients. This experiment operates in a steady state with stationary conditions for tens of seconds, giving excellent statistics for turbulence. The experiment gives a full realization of the cylindrical sheared slab plasma model. 

Large--scale neutral gas wave motions have a significant influence on energy balance in the Earth's atmospheric circulation [\emph{Pedlosky} (1987), \emph{Satoh} (2004)]. However, the presence of charged particles in the electrically--conductive weakly--ionized ionosphere substantially enriches the conditions for the propagation of low--frequency wave modes that are different in nature. Numerous ground--based and satellite observations [\emph{Cavalieri, et al.} (1974), \emph{Cavalieri} (1976), \emph{Manson, et al.} (1981), \emph{Hirooka and Hirota} (1985), \emph{Randel} (1987), \emph{Sorokin} (1988), \emph{Sharadze, et al.} (1988, 1989), \emph{Williams and Avery} (1992), \emph{Forbes and Leveroni} (1992), \emph{Bauer, et al.} (1995), \emph{Zhou, et al.} (1997), \emph{Lastovicka} (1997), \emph{Smith} (1997), \emph{Lawrence and Jarvis} (2003), \emph{Burmaka, et al.} (2006), \emph{Alperovich and Fedorov} (2007), \emph{Fagundes, et al.} (2005)] show that planetary--scale (with wavelength $\lambda>1000\,$km and periods of several days) wave perturbations of Electromagnetic (EM) origin regularly exist in different ionospheric layers. Increasing interest in planetary--scale Ultra Low Frequency (ULF) wave perturbations is caused by the fact that many ionospheric phenomena from the same frequency range play the role of ionospheric precursors of some extraordinary phenomena (earthquakes, volcano eruptions, etc.) [\emph{Hajkowicz} (1991), \emph{Liperovsky, et al.} (1992), \emph{Cheng and Huang} (1992)] and also appear as the ionospheric response to the anthropogenic activity [\emph{Pokhotelov, et al.} (1995), \emph{Shaefer, et al.} (1990), \emph{Burmaka and Chernogor} (2004), \emph{Burmaka, et al.} (2005)]. Forced oscillations of that kind, under the impulsive impacts on the ionosphere and during magnetospheric storms, were also observed [\emph{Hajkowicz} (1991)]. 

In recent years, an increasing number of theoretical and experimental investigations have been devoted to the investigation of the dynamics of Rossby waves (induced by the spatial inhomogeneity of the Coriolis parameter) in the Earth's ionosphere. \emph{Dokuchaev} (1959) first indicated the necessity of accounting for the interaction of an induced electric current with the Earth's magnetic field on the wind's dynamics. The next step was done by \emph{Tolstoy} (1967), who pointed out the importance of global factor acting permanently in the ionosphere --- the space inhomogeneity of the geomagnetic field on the dynamics of Rossby type waves in the Earth's ionospheric E--layer. The waves were entitled Hydromagnetic Gradient (HMG) waves. It was shown that HMG waves can couple with the Rossby waves in the E--layer height ionosphere. Tolstoy suggested that HMG waves may appear as traveling perturbations of the Sq current system producing from a few to several tenths of nT strong variations of the geomagnetic field. 

A review of the mechanisms for the generation of zonal flows and magnetic field fluctuations by various coupled EM ULF waves in the Earth's ionospheric E--layer on the base of our investigations [\emph{Kaladze, et al.} (2012a,b, 2013a,b)] are described in \emph{Kahlon and Kaladze} (2015). Taking into the account a latitudinal inhomogeneity of Coriolis parameter and geomagnetic field propagation of Coupled Internal Gravity Alfv\'en (CIGA), Coupled Rossby--Khantadze (CRK) and Coupled Rossby--Alfv\'en--Khantadze (CRAK) waves is revealed and studied. In \emph{Kahlon and Kaladze} (2015) show that the instability of short wavelength turbulence of such coupled waves may lead to the excitation of low--frequency and large--scale perturbation of the sheared zonal flow and sheared magnetic field. The nonlinear mechanism of the instability is based on the parametric triple interaction of finite amplitude coupled waves leading to the inverse energy cascade toward the longer wavelengths. The possibility of generation of the intense mean magnetic fields is shown. Obtained growth rates are discussed for each case of the considered coupled waves.

\subsection*{The objectives of the Helimak plasma project}

The main goal of the rotating Helimak project is to develop nonlinear equations for the formation of vortical structures in the Helimak and for mesoscale cyclones and anticyclones in the ionosphere, incorporating all features mentioned above. Here the new research relates to understanding the generation of coherent magnetic plasma structures and turbulence in the Helimak and their correspondence to plasma structures in the ionosphere. A part of the project will be dedicated to interpretation of laboratory, numerical, and observational studies of these structures in Helimak and their relation to structures in the ionosphere. Particular attention will be paid to the analysis of the ionospheric responses to the man--made activities and extraordinary natural phenomena (earthquake, storms, hurricanes etc.).

\subsection*{The main objectives of the Helimak--ionosphere modeling project}
\begin{itemize}
  \item To perform a theoretical investigation of the generation of magnetic coherent structures and turbulence in Helimak at The University of Texas and their corresponding structures in the ionosphere. 
  \item To study the similarities between the coherent structures in Helimak and ionospheric plasmas. The dynamics of these structures in Helimak and ionospheric plasma is developed. 
  \item To provide a novel mechanism for the formation of vortical structures in the Helimak and for mesoscale cyclones and anticyclones in the ionosphere. 
  \item To elucidate the role of the charged particles in the nonlinear dynamics of magnetic waves and drift waves. 
  \item To apply the developed theory to the physics of laboratory and ionospheric atmospheres.
\end{itemize}  

%7.5.4
\subsection{Helicon waves and LHCD for controlling the toroidal currents}

\subsection*{Helicon RF plasma waves and lower hybrid current drive}

Antennas are designed to drive helical RF plasma waves to maintain the toroidal plasma currents in steady state or long pulse tokamaks. The method is used in the DIII--D tokamak plasmas to control the toroidal plasma current profiles and the current strength, with the aim of achieving a degree of confirmation between the theoretical models and the DIII--D plasma data. The Pinkser team developed 12 element Helicon antennas that are currently in DIII--D. 

Plans are to verify modeling of the helicon wave antennas as carried out earlier for the higher--frequency LHCD--driven plasmas in the Tore Supra tokamak for the theory and simulations of the interactions of the RF waves in turbulent plasma. The plasma inhomogeneity produces RF mode conversions. The Tore Supra team published [\emph{Horton, et al.} (2013)] a modeled and successful experiment maintaining long--time plasma currents driving the fast RF waves by a complex $3\,$GHz RF antenna. The slow LHCD waves and fast wave helicon waves each have advantages for steady--state fusion plasmas. The RF simulation procedures are similar for a given turbulent tokamak plasma; however, the LHCD is smaller wavelength and easier to maintain than the large fast wave antennas. Currently, validation of the fast--wave current drive model is taking place in the DIII--D plasmas. The tokamak community now has two RF methods to maintain steady--state toroidal plasma currents. Drift wave turbulence couples the two types --- slow and fast --- RF waves and needs to be taken into account in the plasma data analysis and modeling of RF wave driven plasmas.

\subsection*{Lower Hybrid Current Drive (LHCD) in EAST Tokamak}

The EAST tokamak has successfully driven long--time steady--state tokamak plasma currents. Other RF--driven toroidal fusion plasmas include HL--2A and KSTAR machines that use Klystrons and Gyrotrons to maintain steady--state fusion plasmas. The comparisons of the results from EAST, KSTAR, WEST and DIII--D is a high priority in designing a steady--state ITER toroidal plasma. 

The Department of Energy launched an emphasis on research for ``Long Pulse Tokamak Research" to support and explore critical science and technology for long duration plasma discharges in ITER in 2017. It was stated that the specific areas of interest were to find and make achieving long--pulse high--performance plasmas. Then the DoE numerated these items of concern: (i) studying and developing high--performance operating plasma states that can be robustly produced, sustained, and controlled for long periods of time; (ii) establishing plasma confinement, stability and operational boundaries, (iii) understanding plasma core, pedestal, boundary, and scrape--off layer physics and (iv) developing and demonstrating divertor solutions that provide improved power handling.

%7.5.5
\subsection{Anomalous alpha particle plasma heating rates}

Important parameters in the energy balance of thermonuclear plasmas are the alpha particle heating rates and the division of the alpha particle energy transferred between ions and electrons. Here we review the collective mechanisms for the transfer of the alpha particle energy to the plasma and point out that these collective wave transfer mechanisms are much faster than the collisional transfer rates. The collisional alpha particle $m_a, E_\alpha$ slowing--down time is $\tau_s\simeq 3 m_\alpha m_e v^3_e/[(2\pi)^{1/2} n_e z^2_\alpha e^4\ell n\, \Lambda$] and is of order one second for $n_e = 10^{14}\rm cm^{-3}$ and $T_e\approx \rm 10-20\,$KeV. However, we show that for alpha particle energy distributions peaked around the source energy of $2.5\,$MeV there are electromagnetic instabilities with scale $k^{-1}\simeq c/\omega_{pi}$ and electrostatic instabilities with scale $k^{-1}\gtsim v_i/\omega_{ci}=\rho_i$ that grow on the ion cyclotron period time scale and lead to anomalous diffusion equations for the particle distribution functions. We review results for the electrostatic modes and the electromagnetic modes in \emph{Horton, et al.} (2014).

\begin{thebibliography}{100}

\bibitem{}
Alperovich, L. S. and Fedorov, E. N. (2007). {\it Hydromagnetic Waves in the Magnetosphere and the Ionosphere}, (Dordrecht: Springer, 2007), \url{https://ezproxy.cern.ch/login?url=http://dx.doi./org/10.1007/978-1-4020-6637-5}.

\bibitem{}
Bauer, T. M., Baumjohann, W., Treumann, R. A., Sckopke, N. and L\"uhr, H. (1995). Low--frequency waves in the near--Earth plasma sheet, {\it J. Geophys. Res.} \textbf{100}, pp.~9605--9617, \url{https://doi.org/10.1029/95JA00136}.

\bibitem{}
Bohm, D., Burhop, E. H. S. and Massey, H. S. W. (1949). The use of probes for plasma exploration in strong magnetic fields,  \emph{The Characteristics of Electrical Discharges in Magnetic Field} eds.~A. Guthrie and R. K. Wakerling, (New York, McGraw--Hill).

\bibitem{}
Burmaka, V. P. and Chernogor, L. F. (2004). Clustered instrument studies of ionospheric wave disturbances accompanying rocket launches against the background of nonstationary natural processes, 
{\it Geomag. Aeron.} \textbf{44}, pp.~518--534.

 \bibitem{}
Burmaka, V. P., Taran, L. F. and Chernogor, L. F. (2005). Results of investigations of the wave disturbances in the ionosphere by noncoherent scattering, {\it Adv. Mod. Radiophys.} \textbf{3}, pp.~4--35.

\bibitem{}
Burmaka, V. P., Lysenko, V. N., Chernogor, L. F. and Chernyak, Yu. V. (2006). Wave--like processes in the ionospheric $F$--region that accompanied rocket launches from the Baikonur site, {\it Geomag. Aeron.} \textbf{46}, pp.~742--759, \url{https://doi.org/10.1134/S0016793206060107}.

\bibitem{}
Cavalieri, D. J., Deland, R. J., Poterna, J. A. and Gavin, R. F. (1974). The correlation of VLF propagation variations with atmospheric planetary--scale waves {\it J. Atmos. Terr. Phys.} \textbf{36}, pp.~561--574, \url{https://doi.org/10.1016/0021-9169(74)90082-8}.

\bibitem{}
Cavalieri, D. J. (1976). Traveling planetary--scale waves in the $E$--region, {\it J. Atmos. Terr. Phys.} \textbf{38}, pp.~965--977, \url{https://doi.org/10.1016/0021-9169(76)90079-9.}

\bibitem{}
Chen, F. F. (1965a). Resistive over--stabilities and anomalous diffusion, \emph{Phys. Fluids} \textbf{8}, 
p.~912, \url{http://dx.doi.org/10.1063/1.1761335}.

\bibitem{}
Chen, F. F. (1965b). Excitation of drift instabilities in thermionic plasmas, \emph{J. Nucl. Energy} Pt. C, 
p.~399, \url{http://dx.doi.org/10.1088/0368-3281/7/4/305}.

\bibitem{}
Chen, F. F. (1966). Microinstability and shear stabilization of a low--$\beta$ rotating, resistive plasma, {\it Phys. Fluids} \textbf{9}, p.~965, \url{https://doi.org/10.1063/1.1761798}.

\bibitem{}
Chen, F. F. (1967). Nonlocal drift modes in cylindrical geometry, {\it Phys. Fluids} \textbf{10}, p.~1647, \url{https://doi.org/10.1063/1.1762340}.

\bibitem{}
Chen, L. and Cheng, C. Z. (1980). Drift--wave eigenmodes in toroidal plasmas, \emph{Phys. Fluids} \textbf{23}, p.~2242, \url{https://doi.org/10.1063/1.862907}.

\bibitem{}
Cheng, K.  and Huang, Y--N. (1992). Ionospheric disturbances observed during the period of Mount Pinatubo eruptions in June 1991, \emph{J. Geophys. Res.} \textbf{97}, pp.~16995--7004,
\url{https://doi.org/10.1029/92JA01462}.

\bibitem{}
Choi, D. I. and Horton, W. (1980). Weakly localized two--dimensional drift modes, \emph{Phys. Fluids} \textbf{23}, p.~356, \url{https://doi.org/10.1063/1.862980}.

\bibitem{}
D'Angelo, N. and Motley, R. W. (1963). Low--frequency oscillations in a potassium plasma, \emph{Phys. Fluids} 
\textbf{6}, p.~422, \url{http://dx.doi.org/10.1063/1.1706749}. 

\bibitem{}
Dokuchaev, V.P. (1959). P. Influence of the earth's magnetic field on the ionospheric winds, \emph{Izvestia  AN SSSR Seria Geophysica} \textbf{5}, pp.~783--787, \url{}.

\bibitem{}
 Fagundes, P. R.,Pillat, V. G., Bolzan, M. J. A., Sahai, Y., Becker--Guedes, F., Abalde, J. R., Aranha, S. L. and Bittencourt, J. A. (2005). Observations of F--layer electron density profiles modulated by planetary wave type oscillations in the equatorial ionospheric anomaly region, \emph{J. Geophys. Res.} \textbf{110}, p.~A12302, \url{https://doi.org/10.1029/2005JA011115}.

\bibitem{}
Forbes, J. and Leveroni, S. (1992). Quasi 16--day oscillation in the ionosphere, {\it Geophys. Res. Lett.} \textbf{19}, pp.~981--984, \url{https://doi.org/10.1029/92GL00399}.

\bibitem{}
Frieman, E. A., Rewoldt, G., Tang, W. M. and Glasser, A. H. (1980). General theory of kinetic ballooning modes, \emph{Phys. Fluids} \textbf{23}, p.~1750, \url{https://doi.org/10.1063/1.863201}.

\bibitem{}
Hastie, R. J., Hesketh, K. W. and Taylor, J. B. (1979). Shear damping of two--dimensional drift waves in a large--aspect--ratio tokamak, \emph{Nucl. Fusion} \textbf{19}, p.~1223, \url{https://doi.org/10.1088/0029-5515/19/9/006}.

\bibitem{}
Hajkowicz, L. A. (1991). Global onset and propagation of large--scale traveling ionospheric disturbances as a result of the great storm of 13 March 1989, \emph{Planet. Space Sci.} \textbf{39}, pp.~583--593, \url{https://doi.org/10.1016/0032-0633(91)90053-D}.

\bibitem{}
Hendel, H. W., Chu, T. K. and Politzer, P. A. (1968). Collisional drift waves --- Identification, stabilization, and enhanced plasma transport, \emph{Phys. Fluids} \textbf{11}, p.~2426, \url{http://dx.doi.org/10.1063/1.1691833}.

\bibitem{}
Hinton, F. L. and Horton, Jr., C. W. (1971). Amplitude limitation of a collisional drift wave instability, \emph{Phys. Fluids} \textbf{14}, pp.~116--123, \url{http://dx.doi.org/10.1063/1.1693260}.

\bibitem{}
Hirooka, T. and Hirota, I. (1985). Normal mode Rossby waves observed in the upper stratosphere. Part II: Second antisymmetric and symmetric modes of zonal wavenumbers 1 and 2, {\it J. Atmos. Sci.} \textbf{42}, 
pp.~536--548, \url{https://doi.org/10.1175/1520-0469(1985)042<0536:NMRWOI>2.0.CO;2}.

\bibitem{}
Horton, W., Estes, R., Kwak, H. and Choi, D--I. (1978). Toroidal mode coupling effects on drift wave stability, \emph{Phys. Fluids} \textbf{21}, p.~1366, \url{https://doi.org/10.1063/1.862378}.

\bibitem{}
Horton, W., Choi, D--I. and Tang, W. M. (1981). Toroidal drift modes driven by ion pressure gradients, \emph{Phys. Fluids} \textbf{24}, pp.~1077--1085, \url{https://doi.org/10.1063/1.863486}.

\bibitem{}
Horton, W. (1999). Drift waves and transport, \emph{Rev. Mod. Phys.} \textbf{71}, p.~735, \url{http://dx.doi.org/10.1103/RevModPhys.71.735}.

\bibitem{}
Horton, W., Goniche, M., Peysson, Y., Decker, J., Ekedahl, A. and Litaudon, X. (2013). Penetration of lower--hybrid current drive waves in tokamaks, \emph{Phys. Plasmas} \textbf{20}, p.~112508, \url{http://dx.doi.org/10.1063/1.4831981}.

\bibitem{}
Kahlon, L. Z. and Kaladze, T. D. (2015). Generation of zonal flow and magnetic field in the ionospheric E--layer, \emph{J. Plasma Phys.} \textbf{81}, p.~905810512, \url{https://doi.org/10.1017/S002237781500080X}.

\bibitem{}
Kaladze, T. D., Kahlon, L. Z. and Tsamalashvili, L. V. (2012). Excitation of zonal flow and magnetic field by Rossby--Khantadze electromagnetic planetary waves in the ionospheric E--layer, \emph{Phys. Plasmas} \textbf{19}, 
p.~022902, \url{https://doi.org/10.1063/1.3681370}.

\bibitem{}
Kaladze, T. D., Kahlon, L. Z., Tsamalashvili, L. V. and Kaladze, D. T. (2012b). Generation of zonal flow and magnetic field by coupled internal--gravity and Alfv\'en waves in the ionospheric E--layer, \emph{J. Atmos. Sol.: Terres. Phys.}
\textbf{89}, pp.~110--119, \url{https://doi.org/10.1016/j.jastp.2012.08.012}.

\bibitem{}
Kaladze, T. D., Horton, W., Kahlon, L. Z., Pokhotelov, O. and Onishshenko, O. (2013). Generation of zonal flow and magnetic field by coupled Rossby--Alfv\'en--Khantadze waves in the Earth's ionospheric E--layer, \emph{Phys. Scr.} \textbf{88}, p.~065501, \url{https://doi.org/10.1088/ 0031-8949/88/06/065501}.

\bibitem{}
Kaladze, T. D., Horton, W., Kahlon, L. Z., Pokhotelov, O. and Onishshenko, O. (2013b). Zonal flows and magnetic fields driven by large--amplitude Rossby Alfv\'en Khantadze waves in the E--layer ionosphere, \emph{J. Geophys. Res: Space Phys.} \textbf{118}, pp.~7822--7833, \url{https://doi. org/10.1002/2013JA019415}.

\bibitem{}
Lastovicka, J. (1997). Observations of tides and planetary waves in the atmosphere--ionosphere system, \emph{Adv. Space Res.} \textbf{20}, pp.~1209--1222, \url{https://doi.org/10.1016/S0273-1177(97)00774-6}. 

\bibitem{}
Lawrence, A. R. and Jarvis, M. J. (2003). Simultaneous observations of planetary waves from 30 to $220\,$km, {\it J. Atmos. Solar--Terr. Phys.} \textbf{65}, pp.~765--777, \url{https://doi.org/10.1016/S1364-6826(03)00081-6}.

\bibitem{}
Liperovsky, V. A., Pokhotelov, O. A. and Shalimov, S. L. (1992). Ionospheric earthquake precursors (Moscow: Nauka, 1992).

\bibitem{}
Luckhardt, S. (1999). The Helimak: A one dimensional toroidal plasma system, Technical Report, University of California, San Diego, 1999; \url{http://orion.ph.utexas.edu/~starpower}.

\bibitem{}
Manson, A. H., Heek, C. E. and Gregory, J. B. (1981). Winds and waves (10 min--30 days) in the mesosphere and lower thermosphere at Saskatoon ($52^\circ$N, $107^\circ$W, L$=4.3$) during the year, October 1979 to July 1980 {\it J. Geophys. Res.} \textbf{86}, pp.~9615--9625, \url{https://doi.org/10.1029/JC086iC10p0961}.

\bibitem{}
M\"uller, S. H., Fasoli, A., Labit, B., McGrath, M., Podesta, M. and Poli, F. M. (2004). Effects of a vertical magnetic field on particle confinement in a magnetized plasma torus, {\it Phys. Rev. Lett.} \textbf{93}, p.~165003, 
\url{https://doi.org/10.1103/PhysRevLett.93.165003}.

\bibitem{}
Parail, V. V., Pereverzev, G. V. and Vojtsekhovich, I. A. (1985) in {\it Proceedings of the IAEA Conference on Plasma Physics and Controlled Thermonuclear Fusion} (London, 1985), Vol.~I, p.~605.

\bibitem{}
Pedlosky, J. (1987). \emph{Geophysical Fluid Dynamics}, 2nd Ed. (Berlin: Springer) \emph{Science}.

\bibitem{}
Pokhotelov, O. A., Parrot, M., Fedorov, E. N., Pilipenko, V. A., Surkov, V. V. and Gladychev. V. A. (1995). Response of the ionosphere to natural and man--made acoustic sources, \emph{Annales Geophysicae} \textbf{13}, pp.~1197--1210, \url{https://doi.org/10.1007/s00585-995-1197-2}.

\bibitem{}
Randel, W. J. (1987). A Study of planetary waves in the southern winter troposphere and stratosphere. Part I: Wave structure and vertical propagation, {\it J. Atmos. Sci.} \textbf{44}, pp.~917--935, \url{https://doi.org/10.1175/1520-0469(1987)044<0917:ASOPWI>2.0.CO;2}.
 
\bibitem{}
Rypdal, K. and Ratynskaia, S. (2005). Onset of turbulence and profile resilience in the Helimak configuration, {\it Phys. Rev. Lett.} \textbf{94}, p.~225002, \url{https://doi.org/10.1103/PhysRevLett.94.225002}.

\bibitem{}
Satoh, M. (2004). {\it Atmospheric Circulation Dynamics and General Circulation Models} (Berlin: Springer) (2004).

\bibitem{}
Shaefer, L. D., Rock, D. R., Lewis, J. P. and Warshaw, S. I. (1999). Detection of explosive events by monitoring acoustically--induced geomagnetic perturbations, Lawrence Livermore Laboratory 94550 (Livermore, CA, 1999), UCRL--ID--133240.

\bibitem{}
Sharadze, Z. S., Dzhaparidze, G. A., Kikvilashvili, G. B., Liadze, Z. L., Matiashvili, T. G. and Mosashvili, N. V. (1988). Wave disturbances of non--acoustic origin in the mid-latitude ionosphere, {\it Geomag. Aeron.} \textbf{28}, pp.~446--451, ISSN:0016-7940.

\bibitem{}
Sharadze, Z. S., Mosashvili, N. V., Pushkova, G. N. and Iudovich, L. A. (1989). Long--period wave disturbances in the ionospheric $\Gamma$--region, {\it Geomag. Aeron.} \textbf{29}, pp.~1032--1035, \url{http://adsabs.harvard.edu/abs/1989IonIs. .46. .123S}.

\bibitem{}
Smith, A. K. (1997). Stationary planetary waves in upper mesospheric winds, {\it J. Atmos. Sci.} \textbf{54}, pp.~2129--2145, \url{https://doi.org/10.1175/1520-0469(1997)054<2129:SPWIUM >2.0.CO;2}.

\bibitem{}
Sorokin, V. M. (1988). Wave processes in the ionosphere associated with the geomagnetic field, Izv. Vyssh. Uchebn. Zaved., Radiofiz.; (USSR) \textbf{31}, pp.~827--836, \url{https://doi.org/10.1007/BF01040013}.

\bibitem{}
Tolstoy, I. (1967). Hydromagnetic gradient waves in the ionosphere, \emph{J. Geophys. Res.} \textbf{72}, pp.~1435--1442, \url{https://doi.org/10.1029/JZ072i005p01435}. 

\bibitem{}
Williams, C. R. and Avery, S. K. (1992). Analysis of long--period waves using the mesosphere--stratosphere--troposphere radar at Poker Flat, Alaska, {\it J. Geophys. Res.} \textbf{97}, 
p.~20,855--20,861, \url{https://doi.org/10.1029/92JD02052}.

\bibitem{}
Zhou, Q. H., Sulzer, M. P. and Tepley, C.A. (1997). An analysis of tidal and planetary waves in the neutral winds and temperature observed at low--latitude $E$--region heights, {\it J. Geophys. Res.} \textbf{102}, pp.~11491--11505, 
\url{https://doi.org/10.1029/97JA00440}.

\bibitem{}
Zimmerman, E. D. and Luckhardt, S. C. (1993). Measurement of the correlation spectrum of electrostatic potential fluctuations in an ECRH Helimak plasma, {\it J. Fusion Energy} \textbf{12}, pp.~289--293, \url{https://doi.org/10.1007/BF01079672}.

\end{thebibliography}
\end{document}
